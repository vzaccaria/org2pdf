\documentclass[ignorenonframetext,,aspectratio=169]{beamer}
%% \setbeamertemplate{caption}[numbered]
%% \setbeamertemplate{caption label separator}{:}
%% \setbeamercolor{caption name}{fg=normal text.fg}

%%                    ██                        
%%                   ░██                        
%%   █████   ██████  ░██  ██████  ██████  ██████
%%  ██░░░██ ██░░░░██ ░██ ██░░░░██░░██░░█ ██░░░░ 
%% ░██  ░░ ░██   ░██ ░██░██   ░██ ░██ ░ ░░█████ 
%% ░██   ██░██   ░██ ░██░██   ░██ ░██    ░░░░░██
%% ░░█████ ░░██████  ███░░██████ ░███    ██████ 
%%  ░░░░░   ░░░░░░  ░░░  ░░░░░░  ░░░    ░░░░░░

\definecolor{gray}{RGB}{155,155,155}
\definecolor{gray}{RGB}{155,155,155}

\setbeamercolor{titlelike}{fg=black}
\setbeamercolor{framesubtitle}{fg=black}
\setbeamercolor{subtitle}{fg=gray}
\setbeamercolor{institute}{fg=gray}

\definecolor{links}{HTML}{2A1B81}
\hypersetup{colorlinks,linkcolor=,urlcolor=links}

%% \setbeamercolor{itemize item}{fg=black}
%% \setbeamercolor{normal text}{fg=black,bg=white}
\setbeamercolor{item}{fg=black} % color of bullets
\setbeamercolor{subitem}{fg=gray}
\setbeamercolor{itemize/enumerate subbody}{fg=gray}
\setbeamertemplate{itemize subitem}{{\textendash}}
\setbeamerfont{itemize/enumerate subbody}{size=\footnotesize}
\setbeamerfont{itemize/enumerate subitem}{size=\footnotesize}

% Slide number

\setbeamertemplate{footline}{%
    \raisebox{5pt}{\makebox[\paperwidth]{\hfill\makebox[20pt]{\color{gray}
          \tiny\insertframenumber}}}\hspace*{5pt}}

\usepackage{amssymb,amsmath}
\usepackage{ifxetex,ifluatex}
\usepackage{fixltx2e} % provides \textsubscript
\usepackage{lmodern}
\ifxetex
  \usepackage{fontspec,xltxtra,xunicode}
  \newcommand{\euro}{€}
  \setmainfont{Fira Sans}
  \setsansfont{Fira Sans}
  \setmonofont[Scale=0.8,Ligatures=TeX]{Hasklig Medium}
  \setbeamerfont{title}     {family=\fontspec{Fira Sans Light},shape=\scshape}
  \setbeamerfont{frametitle}{family=\fontspec{Fira Sans Light},shape=\scshape}
  \setbeamertemplate{frametitle}[default][center]

\else
  \ifluatex
    \usepackage{fontspec}
    \defaultfontfeatures{Mapping=tex-text,Scale=MatchLowercase}
    \newcommand{\euro}{€}
  \else
    \usepackage[T1]{fontenc}
    \usepackage[utf8]{inputenc}
      \fi
\fi
% use upquote if available, for straight quotes in verbatim environments
\IfFileExists{upquote.sty}{\usepackage{upquote}}{}
% use microtype if available
\IfFileExists{microtype.sty}{\usepackage{microtype}}{}
\usepackage{url}

% Comment these out if you don't want a slide with just the
% part/section/subsection/subsubsection title:
\AtBeginPart{
  \let\insertpartnumber\relax
  \let\partname\relax
  \frame{\partpage}
}
\AtBeginSection{
  \let\insertsectionnumber\relax
  \let\sectionname\relax
  \frame{\sectionpage}
}
\AtBeginSubsection{
  \let\insertsubsectionnumber\relax
  \let\subsectionname\relax
  \frame{\subsectionpage}
}

\setlength{\parindent}{0pt}
\setlength{\parskip}{6pt plus 2pt minus 1pt}
\setlength{\emergencystretch}{3em}  % prevent overfull lines
\providecommand{\tightlist}{%
  \setlength{\itemsep}{0pt}\setlength{\parskip}{0pt}}
\setcounter{secnumdepth}{0}

\title{synthesis of fixed point code}
\author{Vittorio Zaccaria}
\date{\textless{}2015-07-26 Sun\textgreater{}}

\begin{document}
\frame{\titlepage}

\begin{frame}{why we need fixed point}

\begin{itemize}
\item
  Express fractional numbers, using a fixed number of bits
  \url{http://www.digitalsignallabs.com/fp.pdf}
\item
  Useful for systems without floating-point hardware (DSPs, FPGAs, and
  expensive custom ASIC, microcontrollers).
\item
  Also useful when you need speed, as integer operations are faster than
  floating point ones.
\end{itemize}

\end{frame}

\begin{frame}{Integer representation}

\begin{itemize}
\item
  The formula for calculating the integer representation \(x\) in a
  \(Q_{m.n}\) format of a float number \(f\) is (ah):

  \[x = round(f * 2^n)\]
\item
  To convert it back the following formula is used:

  \[f = x * 2^{-n}\]
\end{itemize}

\end{frame}

\begin{frame}{Range and resolution}

\begin{itemize}
\item
  For a given \(Q_{m.n}\) we have:

  \begin{itemize}
  \item
    its range is \[[ - (2^m) , 2^m -2^{-n}]\]
  \item
    its resolution is \[2^{-n}\]
  \end{itemize}
\end{itemize}

\end{frame}

\begin{frame}{The problem}

\begin{itemize}
\tightlist
\item
  operations can produce results that have more bits than the operands
  (\textbf{e.g.} multiplications)
\end{itemize}

\[Q_{1.7} * Q_{1.7} \rightarrow Q_{2.14}\]

\begin{itemize}
\tightlist
\item
  Solution: temporarily use a bigger register but then truncate or round
  back to \(Q_{1.7}\)
\end{itemize}

\end{frame}

\begin{frame}{Ada approach}

\begin{itemize}
\item
  Ada uses a delta type; with this you can specify what is the minimum
  difference between two float numbers to be considered different.
\item
  The compiler however will choose \(2^{-12} = \frac{1}{4096}\), not
  \(\frac{1}{3600}\).
\end{itemize}

\end{frame}

\begin{frame}{Concepts}

Two's complement

\begin{itemize}
\tightlist
\item
  Assume \(x\) is \(N\) bits. We define the complement \(x_2\) such
  that:
\end{itemize}

\(x_2 + x = 2^N\)

\begin{itemize}
\tightlist
\item
  \(2^N\) is represented by zeroes over \(N\) bits and a 1 outside the
  \(N\) bits. So, if we focus on the lower \(N\) bit, we'll see that
  \(x_2\) behaves like a real \textbf{inverse} of \(x\) in `+'.
\end{itemize}

\end{frame}

\end{document}
