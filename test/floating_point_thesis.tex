% Created 2015-10-26 Mon 10:25
\documentclass[bigger]{beamer}
\author{Vittorio Zaccaria}
\date{\textit{<2015-07-26 Sun>}}
\title{synthesis of fixed point code}
\hypersetup{
  pdfkeywords={},
  pdfsubject={},
  pdfcreator={Emacs 24.5.1 (Org mode 8.2.10)}}
\begin{document}

\maketitle
\tableofcontents


\section{why we need fixed point}
\label{sec-1}

\begin{itemize}
\item Express fractional numbers, using a fixed number of bits \url{http://www.digitalsignallabs.com/fp.pdf}

\item Useful for systems without floating-point hardware (DSPs, FPGAs, and expensive custom ASIC, microcontrollers).

\item Also useful when you need speed, as integer operations are faster than floating point ones.
\end{itemize}

\section{Integer representation}
\label{sec-2}

\begin{itemize}
\item The formula for calculating the integer representation $x$ in a $Q_{m.n}$ format of a float number $f$ is (ah):

$$x = round(f * 2^n)$$

\item To convert it back the following formula is used:

$$f = x * 2^{-n}$$
\end{itemize}

\section{Range and resolution}
\label{sec-3}

\begin{itemize}
\item For a given $Q_{m.n}$ we have:

\begin{itemize}
\item its range is $$[ - (2^m) , 2^m -2^{-n}]$$

\item its resolution is $$2^{-n}$$
\end{itemize}
\end{itemize}

\section{The problem}
\label{sec-4}

\begin{itemize}
\item operations can produce results that have more bits than the operands (\textbf{e.g.} multiplications)
\end{itemize}

$$Q_{1.7} * Q_{1.7} \rightarrow Q_{2.14}$$

\begin{itemize}
\item Solution: temporarily use a bigger register but then truncate or round back to $Q_{1.7}$
\end{itemize}

\section{Ada approach}
\label{sec-5}

\begin{itemize}
\item Ada uses a delta type; with this you can specify what is the minimum difference between two float numbers to be
considered different.
\end{itemize}


\begin{itemize}
\item The compiler however will choose $2^{-12} = \frac{1}{4096}$, not $\frac{1}{3600}$.
\end{itemize}

\section{Concepts}
\label{sec-6}

Two's complement

\begin{itemize}
\item Assume $x$ is $N$ bits. We define the complement $x_2$ such that:
\end{itemize}

$x_2 + x = 2^N$

\begin{itemize}
\item $2^N$ is represented by zeroes over $N$ bits and a 1 outside the $N$ bits. So, if we focus on the lower $N$ bit,
we'll see that $x_2$ behaves like a real \textbf{inverse} of $x$ in '+'.
\end{itemize}
% Emacs 24.5.1 (Org mode 8.2.10)
\end{document}
